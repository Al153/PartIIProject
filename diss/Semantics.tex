\documentclass[twoside,a4paper,11pt]{article}
\usepackage{amsmath}
\usepackage[a4paper,includeheadfoot,margin=2.54cm]{geometry}
\usepackage{breqn}
\usepackage{stmaryrd}
\usepackage{amssymb}


\newcommand{\db}[1]{{\bf [\![}#1{\bf ]\!]}}
\newcommand{\deno}[1]{\db{#1}(v)}
\newcommand{\setComp}[2]{\left\lbrace #1 \mid #2 \right\rbrace}
\newcommand{\clos}[0]{closure(A, v)}
\newcommand{\typeRule}[2]{\Sigma\vdash #1 \colon #2}
\newcommand{\denoRule}[2]{#1 \in \deno{#2}}
\newcommand{\opRule}[3]{#1 \triangleleft_{#2, v} #3}

\newcommand{\phiRule}[3]{\Phi(\Sigma, #1, #2, #3)}
\newcommand{\psiRule}[2]{\Psi(\Sigma, #1, #2)}

\begin{document}

\section{Semantics}
\subsection{Grammar and definitions}

\paragraph{FindPair queries}



\begin{equation}
\label{PDefifinition}
\begin{split}
P  &\rightarrow Rel(R) \mbox{ Find pairs related by the relation R}\\
&\mid RevRel(R) \mbox{ Find pairs related by the relation R in the reverse direction}\\
&\mid Chain(P, P) \mbox{   Find pairs related by the first subquery followed by the second}\\
&\mid And(P, P) \mbox{  Find pairs related by both of the sub-queries}\\
&\mid AndRight(P, S) \mbox{  Find pairs related by P where the right value is a result of s}\\
&\mid AndLeft(P, S) \mbox{  Find pairs related by P where the left value is a result of s}\\
&\mid Or(P, P) \mbox{  Find pairs related by either of the sub-queries}\\
&\mid Distinct(P) \mbox{  Find pairs related by P that are not symmetrical}\\
&\mid Id_A \mbox{ Identity relation}\\
&\mid Exactly(\mathit{n}, P) \mbox{  Find pairs related by n repetitions of P}\\
&\mid Upto(\mathit{n}, P) \mbox{  Find pairs related by upto n repetitions of P}\\
&\mid FixedPoint(P) \mbox{  Find the transitive closure of P}\\
\end{split}
\end{equation} 


where $n$ denotes a natural number. These queries lookup a set of pairs of obejcts.\\
\paragraph{FindSingle queries}
\begin{equation}
\label{PDefifinition}
\begin{split}
S & \rightarrow Find(F) \mbox{ Find values that match the findable F}\\
&\mid From(S, P) \mbox{ Find values that are reable from results of S via P}\\
&\mid AndS(S, S') \mbox{ Find values that are results of both subqueries}\\
& \mid OrS(S, S') \mbox{ Find values that are results of either subquery}
\end{split}
\end{equation} 
Which lookup sets of single objects.\\


\paragraph{Object Types}

$$ \tau \rightarrow A \mid B \mid C \mid .. $$

Which are the ``real world'' types stored in the database. These correspond to the user's scala classes.

\paragraph{Named relations}

$$R \rightarrow r_1 \mid r_2 \mid .. $$

\paragraph{Findables}

$$F_A \rightarrow f_1 \mid f_2 \mid ... $$

which are names for defined partial functions

$$f \colon A \rightharpoonup \{True, False\} $$

For some given object type A. (ie a findable is an index)

\paragraph{A schema} $\Sigma$ is made up of three partial functions:

$$\Sigma_{rel}\colon R \rightharpoonup \tau\times\tau $$ 
$$\Sigma_{findable}\colon F \rightharpoonup \tau $$
$$\Sigma_{table}\colon \tau \rightharpoonup \{True, False\} $$ 

Which, give the types of relations and findables, and validiate the existence of a type. When it is obvious from the context, I shall use simply use $\Sigma(x)$ to signify application of the appropriate function.
 
\paragraph{A view} $v \in V_\Sigma$, for a given schema represents an immutable state of a database.

It represents a pair of partial functions. Firstly the named-relation lookup function

\begin{equation}
v \in V_\Sigma \Rightarrow v_{rel}(r) \in \wp(A \times B)\mbox{ if $\Sigma(r)\downarrow (A, B)$}
\end{equation} 

That is, if a relation r is in the schema, then $v(r)$ is a set of pairs of objects with object type $\Sigma(r)$. Here, and from this point onwards I am using $\wp(s)$ to represent the powerset of a set, and $f(x) \downarrow y$ to mean $f$ is defined at $x$ and $f(x)=y$


The next function of a view is the type-lookup function, it looks up a findable and returns

\begin{equation}
v \in V_\Sigma \Rightarrow v_{table}(A) \in \wp(A)\indent \mbox{if $\Sigma(A)\downarrow True$}
\end{equation} 

That is, $v(A)$ is a set of objects of type $A$ stored in the view, and $A$ is a member of the schema $\Sigma$. Again I shall overload these two functions where it is clear from the context which is to be used.

\subsection{typing}

Typing rules take two forms. Firstly typing of pair queries:
$$ \Sigma \vdash P\colon (A, B)$$

Which means ``under the schema $\Sigma$, pair query $P$ returns a subset of $A \times B$''.  The second is for single queries:

$$ \Sigma \vdash S \colon A $$

Which means ``under the schema $\Sigma$ single query returns a subset of $A$''

The rules of the first kind are as follows

\[ \begin{array}{c}
\displaystyle\mbox{(Rel)}\frac{\Sigma(r)\downarrow(A, B)}{\typeRule{Rel(r)}{(A, B)}} \\[3ex]

\displaystyle\mbox{(Rev)}\frac{\Sigma(r)\downarrow(B, A)}{\typeRule{Rel(r)}{(A, B)}} \\[3ex]

\displaystyle\mbox{(Id)}\frac{\Sigma(A)\downarrow True}{\typeRule{Id_A}{(A, A)}} \\[3ex]

\displaystyle\mbox{(Chain)}\frac{\typeRule{P}{(A, B)} \indent \typeRule{Q}{(B, C)}}{\typeRule{Chain(P, Q)}{(A, C)}} \\[3ex]

\displaystyle\mbox{(And)}\frac{\typeRule{P}{(A, B)} \indent \typeRule{Q}{(A, B)}}{\typeRule{And(P, Q)}{(A, B)}} \\[3ex]

\displaystyle\mbox{(Or)}\frac{\typeRule{P}{(A, B)} \indent \typeRule{Q}{(A, B)}}{\typeRule{Or(P, Q)}{(A, B)}} \\[3ex]

\displaystyle\mbox{(Distinct)}\frac{\typeRule{P}{(A, B)}}{\typeRule{Distinct(P)}{(A, B)}} \\[3ex]

\displaystyle\mbox{(AndLeft)}\frac{\typeRule{P}{(A, B)} \indent \typeRule{S}{(A)}}{\typeRule{AndLeft(P,S)}{(A, B)}} \\[3ex]

\displaystyle\mbox{(AndRight)}\frac{\typeRule{P}{(A, B)} \indent \typeRule{S}{B}}{\Sigma \vdash \typeRule{AndRight(P, S)}{(A, B)}} \\[3ex]

\displaystyle\mbox{(Exactly)}\frac{\typeRule{P}{(A, A)}}{\typeRule{Exactly(n, P)}{(A, A)}} \\[3ex]

\displaystyle\mbox{(Upto)}\frac{\typeRule{P}{(A, A)}}{\typeRule{Upto(n, P)}{(A, A)}} \\[3ex]

\displaystyle\mbox{(FixedPoint)}\frac{\typeRule{P}{(A, A)}}{\typeRule{FixedPoint(P)}{(A, A)}} \\[3ex]

\end{array} \]


The rules for types of Single queries are similar:

\[ \begin{array}{c}
\displaystyle\mbox{(Find)}\frac{\Sigma(f)\downarrow(A)}{\typeRule{Find(f)}{A}} \\[3ex]

\displaystyle\mbox{(From)}\frac{\typeRule{P}{(A, B)} \indent \typeRule{S}{A}}{\typeRule{From(S, P)}{B}} \\[3ex]

\displaystyle\mbox{(AndS)}\frac{  \typeRule{S}{A} \indent  \typeRule{S'}{A}}{\typeRule{AndS(S, S')}{A}} \\[3ex]

\displaystyle\mbox{(OrS)}\frac{  \typeRule{S}{A} \indent  \typeRule{S'}{A}}{\typeRule{OrS(S, S')}{A}} \\[3ex]
\end{array}
\]


\subsection{Operational Semantics}

Now we shall define a set of rules for determining if a pair of objects is a valid result of a query. We're interested in forming a relation $a \triangleleft p$ to mean``a is a valid result of query Q''. This is dependent on the current view $v: View_{\Sigma}$, and the type of the expression. Hence we define $\opRule{(a, b)}{(A, B)}{P}$ for pair queries $P$ and $\opRule{a}{A}{S}$ for single queries $S$.

\[ \begin{array}{c}
\displaystyle\mbox{(Rel)}\frac{(a,b) \in v(r)}{\opRule{(a, b)}{(A, B)}{Rel(r)}} \\[3ex]

\displaystyle\mbox{(Rev)}\frac{(b,a) \in v(r)}{\opRule{(a, b)}{(A, B)}{RevRel(r)}} \\[3ex]

\displaystyle\mbox{(Id)}\frac{a \in v(A)}{\opRule{(a, a)} {(A, A)} {Id_A}} \\[3ex]

\displaystyle\mbox{(Distinct)}\frac{\opRule{(a,b)}{(A, B)}{P} \indent a \neq b}{\opRule{(a,b)}{(A, B)}{Distinct(P)}} \\[3ex]

\displaystyle\mbox{(And)}\frac{\opRule{(a,b)}{(A, B)}{P} \indent \opRule{(a,b)}{(A, B)}{Q}}{\opRule{(a,b)}{(A, B)}{And(P, Q)}} \\[3ex]

\displaystyle\mbox{(Or1)}\frac{\opRule{(a,b)}{(A, B)}{P}}{\opRule{(a,b)}{(A, B)}{Or(P, Q)}} \\[3ex]

\displaystyle\mbox{(Or2)}\frac{\opRule{(a,b)}{(A, B)}{Q}}{\opRule{(a,b)}{(A, B)}{Or(P, Q)}} \\[3ex]

\displaystyle\mbox{(Chain)}\frac{\opRule{(a,b)}{(A, B)}{P} \indent \opRule{(b,c)}{(B, C)}{Q}}{\opRule{(a,c)}{(A, C)}{Chain(P, Q)}} \\[3ex]

\displaystyle\mbox{(AndLeft)}\frac{\opRule{(a,b)}{(A, B)}{P} \indent \opRule{a}{A}{S}}{\opRule{(a,b)}{(A, B)}{AndLeft(P, S)}} \\[3ex]

\displaystyle\mbox{(AndRight)}\frac{\opRule{(a,b)}{(A, B)}{P} \indent \opRule{b}{
B}{S}}{\opRule{(a,b)}{(A, B)}{AndRight(P, S)}} \\[3ex]

\displaystyle\mbox{(Exactly_0)}\frac{\opRule{(a,b)}{(A, A)}{Id_A}}{\opRule{(a, b)}{(A, A)}{Exactly(0, P)}} \\[3ex]

\displaystyle\mbox{(Exactly_{n+1})}\frac{\opRule{(a,b)}{(A, A)}{P} \indent \opRule{(b, c)}{(A, A)}{Exactly(n, P)}}{\opRule{(a, c)}{(A, A)}{Exactly(n + 1, P)}} \\[3ex]

\displaystyle\mbox{(Upto_0)}\frac{\opRule{(a,b)}{(A, A)}{Id_A}}{\opRule{(a, b)}{(A, A)}{Upto(0, P)}} \\[3ex]

\displaystyle\mbox{(Upto_n)}\frac{\opRule{(a, b)}{(A, A)}{Upto(n, P)}}{\opRule{(a, b)}{(A, A)}{Upto(n + 1, P)}} \\[3ex]

\displaystyle\mbox{(Upto_{n+1})}\frac{\opRule{(a,b)}{(A, A)}{P} \indent \opRule{(b, c)}{(A, A)}{Upto(n, P)}}{\opRule{(a, c)}{(A, A)}{Upto(n + 1, P)}} \\[3ex]

\displaystyle\mbox{(fix1)}\frac{\opRule{(a,b)}{(A, A)}{Id_A}}{\opRule{(a, b)}{(A, A)}{FixedPoint(P)}} \\[3ex]

\displaystyle\mbox{(fix2)}\frac{\opRule{(a,b)}{(A, A)}{P} \indent \opRule{(b, c)}{(A, A)}{FixedPoint(P)}}{\opRule{(a, b)}{(A, A)}{FixedPoint(P)}} \\[3ex]

\end{array} \]

And the FindSingle rules

\[ \begin{array}{c}

\displaystyle\mbox{(Find)}\frac{a \in v(A) \indent f(a)\downarrow True}{\opRule{a}{A}{Find(f)}} \\[3ex]

\displaystyle\mbox{(From)}\frac{\opRule{a}{A}{S} \indent \opRule{(a,b)}{)(A, B)}{P}}{\opRule{b}{A}{From(S, P)}} \\[3ex]

\displaystyle\mbox{(AndS)}\frac{\opRule{a}{A}{S} \indent \opRule{a}{A}{S'}}{\opRule{a}{A}{And(S, S')}} \\[3ex]

\displaystyle\mbox{(OrS1)}\frac{\opRule{a}{A}{S}}{\opRule{a}{A}{Or(S, S')}} \\[3ex]

\displaystyle\mbox{(OrS1)}\frac{\opRule{a}{A}{S'}}{\opRule{a}{A}{Or(S, S')}} \\[3ex]
\end{array} \]



\subsection{Denotational Semantics}

The operational semantics clearly demonstrate membership of a query, but don't give a means to efficiently generate the results of query. To this end, we introduce denotations $\llbracket P \rrbracket$ and $\llbracket S \rrbracket$ such that $$\Sigma \vdash P \colon (A, B) \Rightarrow\llbracket P \rrbracket \colon View_{\Sigma} \rightarrow \wp(A \times B)$$

and $$\Sigma \vdash S \colon A \Rightarrow\llbracket S \rrbracket \colon View_{\Sigma} \rightarrow \wp(A)$$ Such denotations should be compositional and syntax directed, whilst still corresponding to the operational semantics.

\[ \begin{array}{c}
 \deno{Rel(r)} = v(r) \\[3ex]
 \deno{RevRel(r)} = swap(v(r)) \\[3ex]
 \deno{Id_A} = dup(v(A)) \\[3ex]
 \deno{Chain(P, Q)} = join(\deno{P}, \deno{Q}) \\[3ex]
 \deno{And(P, Q)} = \deno{P} \cap \deno{Q} \\[3ex]
 \deno{Or(P, Q)} = \deno{P} \cup \deno{Q} \\[3ex]
 \deno{AndLeft(P, S)} = filterLeft(\deno{P}, \deno{S}) \\[3ex]
 \deno{AndRight(P, S)} = filterRight(\deno{P}, \deno{S}) \\[3ex]
 \deno{Distinct(P)} = distinct(\deno{P}) \\[3ex]
 
 \deno{Exactly(n, P)} = (\lambda pairs. join(\deno{P}, pairs))^n \deno{Id_A} \\[3ex]
 \deno{Upto(n, P)} = (\lambda pairs. join(\deno{P}, pairs) \cup pairs)^n \deno{Id_A} \\[3ex]
 \deno{FixedPoint(P)} = fix (\lambda pairs. join(\deno{P}, pairs) \cup pairs)  \mbox{ in the domain $\clos$}\\[3ex]
\end{array} \]

And similarly with single queries

\[ \begin{array}{c}
\deno{Find(f)} = \setComp{a \in v(A)}{f(a)\downarrow True} \mbox{ for $\Sigma(f) = A$} \\[3ex]

\deno{From(S, P)} =  \setComp{b}{(a, b) \in \deno{P} \wedge a \in \deno{S}}   \\[3ex]

\deno{AndS(S, S')} = \deno{S} \cap \deno{S'} \\[3ex]

\deno{OrS(S, S')} = \deno{S} \cup \deno{S'} \\[3ex]

\end{array}\]

with the following definitions:
$$ swap(s) = \setComp{(b,a)}{(a, b) \in s}$$
$$ dup(s) = \setComp{(a,a)}{a \in s}$$
$$ join(p, q) = \setComp{(a, c)}{ \exists b. (a, b) \in p \wedge (b, c) \in q}$$
$$ distinct(s) = \setComp{(a, b) \in s} { a \neq b} $$
$$ filterLeft(p, s) = \setComp{(a, b) \in p}{a \in s}$$
$$ filterRight(p, s) = \setComp{(a, b) \in p}{b \in s}$$


\subsection{The domain \clos}

For the subsequent proofs it is necessary to define the scott domain $\clos$ for some
object type $A$ and $View_{\Sigma}$ $v$. This domain is the set of subsets $x$ such that $\deno{Id_A} \subseteq x \subseteq A \times A$ with bottom element $\bot = \deno{Id_A}$ and partial order $x \sqsubseteq y \Leftrightarrow x \subseteq y$. From this point on, I shall use $x \subseteq y$ to mean $y \sqsubseteq x$.

\hfill\begin{minipage}{\dimexpr\textwidth-1cm}
\paragraph{Theorem: $\clos$ is a domain}


Firstly, by definition, $$\forall x. x \in \clos \Rightarrow x \supseteq \deno{Id_A}$$ hence $\deno{Id_A}$ is the bottom element.

Secondly for any chain $ x_1 \subseteq x_2 \subseteq x_3 \subseteq ...$, $x_i \in \clos$, there exists a value $\bigsqcup_n x_n \in \clos$ such that $\forall i. \bigsqcup_n x_n \supseteq x_i$ and $\forall y. (\forall i. x_i \subseteq y) \Rightarrow y \supseteq \bigsqcup_n x_n$

\paragraph{proof:}

Take $\bigsqcup_n x_n = \bigcup_n x_n$. This is in $\clos$, since both $\bigcup_n x_n \supseteq \deno{Id_A}$ due to $\forall i. x_i \supseteq \deno{Id_A}$ by definition and $\bigcup_n x_n \subseteq A \times A$, by $$\forall a. (a \in \bigcup_n x_n \wedge \neg (a \in A \times A)) \Rightarrow (\exists i. a \in x_i \wedge \neg a \in A \times A ) \Rightarrow (\exists i. \neg x_i \subseteq A \times A)$$ yielding a contradiction if $\bigcup_n x_n \subseteq A \times A$ does not hold.
$\\ $
We know $\forall i. \bigcup_n x_n \supseteq x_i$ by definition, so it is an upper bound. $\\ $
To prove it is a least upper bound, consider $y$ such that $(\forall i.) y \supseteq x_i$ 

\begin{equation}
\label{LeastUpperBound}
\begin{split}
\forall a. (\exists i. a \in x_i) & \Rightarrow a \in y \\
\forall a. \bigvee_n (a \in x_n) & \Rightarrow a \in y \\
\forall a. (a \in \bigcup_n x_n) & \Rightarrow a \in y \\
\therefore \bigcup_n x_n & \subseteq y\\
\square
\end{split}
\end{equation}

\xdef\tpd{0pt}
\end{minipage}

\prevdepth\tpd

\subsection{Correspondence of operational and denotational semantics}

In order to use the denotational semantics to construct an interpreter or compiler we need to prove they are equivalent to the operational semantics. Namely:

For any pair query $P$, schema $\Sigma$ and $View_{\Sigma}$ $v$:
$$
\typeRule{P}{(A, B)}\Rightarrow \opRule{(a, b}{(A, B)}{P} \Leftrightarrow \denoRule{(a, b)}{P})
$$

And for any single query $S$, schema $\Sigma$ and $View_{\Sigma}$ $v$: 
$$
\typeRule{S}{A}\Rightarrow \opRule{(a}{A}{S} \Leftrightarrow \denoRule{a}{S})
$$

In order to prove these two propositions, we define two induction hypotheses



$$
\phiRule{P}{A}{B} \Leftrightarrow (\typeRule{P}{(A, B)} \Rightarrow \forall v \in View_{\Sigma}, (a, b) \in A \times B. (\opRule{(a, b)}{(A, B)}{P}) \Leftrightarrow \denoRule{(a, b)}{P})))
$$

$$
\psiRule{S}{A} \Leftrightarrow (\typeRule{S}{A} \Rightarrow \forall v \in View_{\Sigma}, a \in A. (\opRule{a}{A}{S}) \Leftrightarrow (\denoRule{a}{S}))
$$

Now we shall induct over the structures of $P$ and $S$ starting with the SingleQuery cases


\paragraph{Case $S = AndS(S', S'')$}
\begin{equation} \label{case AndS}
\begin{split}
\denoRule{a}{S} & \Leftrightarrow a \in (\deno{S'} \cap \deno{S''}) \\
				& \Leftrightarrow (\denoRule{a}{S'} \wedge \denoRule{a}{S''}) \\
				& \Leftrightarrow (\opRule{a}{A}{S'} \wedge \opRule{a}{A}{S''}) \mbox{ by $\psiRule{S'}{A}$, $\psiRule{S''}{A}$}\\
				& \Leftrightarrow (\opRule{a}{A}{AndS(S', S'')} \mbox{ by inversion of (AndS)}
\end{split}
\end{equation}

\paragraph{Case $S = OrS(S', S'')$}
\begin{equation} \label{case OrS}
\begin{split}
\denoRule{a}{S} & \Leftrightarrow a \in (\deno{S'} \cup \deno{S''}) \\
				& \Leftrightarrow (\denoRule{a}{S'} \vee \denoRule{a}{S''}) \\
				& \Leftrightarrow (\opRule{a}{A}{S'} \vee \opRule{a}{A}{S''}) \mbox{ by $\psiRule{S'}{A}$, $\psiRule{S''}{A}$}\\
				& \Leftrightarrow (\opRule{a}{A}{OrS(S', S'')} \mbox{ by inversion of (OrS)}
\end{split}
\end{equation}

\paragraph{Case $S = From(S', P)$}
\begin{equation} \label{case From}
\begin{split}
\denoRule{b}{S} & \Leftrightarrow \exists a \in A. \indent (\denoRule{a}{S'} \wedge \denoRule{(a, b)}{\deno{P}}) \\
				& \Leftrightarrow \exists a \in A. \indent (\opRule{a}{A}{S'} \wedge \opRule{(a, b)}{(A, B)}{P}) \mbox{ by $\psiRule{S'}{A}$, $\phiRule{P}{A}{B}$}\\
				& \Leftrightarrow \opRule{b}{B}{From(S', P)} \indent\mbox{ by inversion of (OrS)}
\end{split}
\end{equation}

\paragraph{Case $S = Find(f)$}
\begin{equation} \label{case Find}
\begin{split}
\denoRule{a}{S} & \Leftrightarrow a \in v(A) \wedge f(a) \downarrow True  \indent \mbox{ by inversion of the type rule (Find)}\\
				& \Leftrightarrow \opRule{a}{A}{Find(f)} \mbox{ by definition}\\
\end{split}
\end{equation}

Now, looking at the FindPair queries

\paragraph{Case $P = Rel(r)$}
\begin{equation} \label{case Rel}
\begin{split}
\denoRule{(a, b)}{P} & \Leftrightarrow (a, b) \in v(r)\\
				& \Leftrightarrow \opRule{(a, b)}{(A, B)}{Rel(r)} \mbox{ by definition}\\
\end{split}
\end{equation}

\paragraph{Case $P = RevRel(r)$}
\begin{equation} \label{case RevRel}
\begin{split}
\denoRule{(a, b)}{P} & \Leftrightarrow (b, a) \in v(r)\\
				& \Leftrightarrow \opRule{(a, b)}{(A, B)}{RevRel(r)} \mbox{ by definition}\\
\end{split}
\end{equation}

\paragraph{Case $P = Id_A$}
\begin{equation} \label{case Id_A}
\begin{split}
\denoRule{(a, b)}{P} & \Leftrightarrow a \in v(A) \wedge a = b\\
				& \Leftrightarrow \opRule{(a, b)}{(A, B)}{Id_A} \mbox{ by definition}\\
\end{split}
\end{equation}

\paragraph{Case $P = Chain(P', Q)$}
$\\ \indent$We have by inversion of the (Chain) type rule $\typeRule{P}{(A, C)} \Leftrightarrow \exists B. \indent\typeRule{P'}{(A, B)} \wedge \typeRule{Q}{(B, C)}$

\begin{equation} \label{case Chain(P', Q)}
\begin{split}
\denoRule{(a, c)}{P} & \Leftrightarrow (a, c) \in join(\deno{P'}, \deno{Q})\\
					& \Leftrightarrow \exists b \in B. \indent \denoRule{(a,b)}{P'} \wedge \denoRule{(b, c)}{Q}\\
					&\Leftrightarrow \exists b \in B. \indent \opRule{(a,b)}{(A, B)}{P'} \wedge \opRule{(b, c)}{(B, C)}{Q}\indent\mbox{by $\phiRule{P'}{A}{B}$, $\phiRule{Q}{B}{C}$} \\
				& \Leftrightarrow \opRule{(a, c)}{(A, C)}{Chain(P', Q)} \mbox{ by definition}\\
\end{split}
\end{equation}

\paragraph{Case $P = And(P', Q)$}
\begin{equation} \label{case And(P', Q)}
\begin{split}
\denoRule{(a, b)}{P} & \Leftrightarrow (a, b) \in (\deno{P'} \cap \deno{Q})\\
					& \Leftrightarrow \denoRule{(a, b)}{P'} \wedge \denoRule{(a, b)}{Q}\\
					&\Leftrightarrow \opRule{(a,b)}{(A, B)}{P'} \wedge \opRule{(a,b)}{(A, B)}{Q}\indent\mbox{by $\phiRule{P'}{A}{B}$, $\phiRule{Q}{A}{B}$} \\
					& \Leftrightarrow \opRule{(a, b)}{(A, B)}{And(P', Q)} \mbox{ by inversion of (And)}\\
\end{split}
\end{equation}

\paragraph{Case $P = Or(P', Q)$}
\begin{equation} \label{case Or(P', Q)}
\begin{split}
\denoRule{(a, b)}{P} & \Leftrightarrow (a, b) \in (\deno{P'} \cup \deno{Q})\\
					& \Leftrightarrow \denoRule{(a, b)}{P'} \vee \denoRule{(a, b)}{Q}\\
					&\Leftrightarrow \opRule{(a,b)}{(A, B)}{P'} \vee \opRule{(a,b)}{(A, B)}{Q}\indent\mbox{by $\phiRule{P'}{A}{B}$, $\phiRule{Q}{A}{B}$} \\
					& \Leftrightarrow \opRule{(a, b)}{(A, B)}{Or(P', Q)} \mbox{ by inversion of (Or1), (Or2)}\\\end{split}
\end{equation}

\paragraph{Case $P = AndLeft(P', S)$}
\begin{equation} \label{case AndLeft(P', Q)}
\begin{split}
\denoRule{(a, b)}{P} & \Leftrightarrow \denoRule{(a, b)}{P'} \wedge \denoRule{a}{S}\\
					&\Leftrightarrow \opRule{(a,b)}{(A, B)}{P'} \wedge \opRule{a}{A}{S}\indent\mbox{by $\phiRule{P'}{A}{B}$, $\psiRule{S}{A}$} \\
					& \Leftrightarrow \opRule{(a, b)}{(A, B)}{AndLeft(P', S)} \mbox{ by inversion of (AndLeft)}\\\end{split}
\end{equation}

\paragraph{Case $P = AndRight(P', S)$}
\begin{equation} \label{case AndRight(P', Q)}
\begin{split}
\denoRule{(a, b)}{P} & \Leftrightarrow \denoRule{(a, b)}{P'} \wedge \denoRule{b}{S}\\
					&\Leftrightarrow \opRule{(a,b)}{(A, B)}{P'} \wedge \opRule{b}{B}{S}\indent\mbox{by $\phiRule{P'}{A}{B}$, $\psiRule{S}{B}$} \\
					& \Leftrightarrow \opRule{(a, b)}{(A, B)}{AndRight(P', S)} \mbox{ by inversion of (AndRight)}\\\end{split}
\end{equation}

\paragraph{Case $P = Distinct(P')$}
\begin{equation} \label{case Distinct(P)}
\begin{split}
\denoRule{(a, b)}{P} & \Leftrightarrow \denoRule{(a, b)}{P'} \wedge a \neq b \\
					&\Leftrightarrow \opRule{(a,b)}{(A, B)}{P'} \wedge a \neq b\indent\mbox{by $\phiRule{P'}{A}{B}$} \\
					& \Leftrightarrow \opRule{(a, b)}{(A, B)}{Distinct(P')} \mbox{ by inversion of (AndRight)}\\\end{split}
\end{equation}

\paragraph{Case $P = Exactly(n, P')$}
$\\ \indent$We have by inversion of the (Exactly) type rule $\typeRule{P}{(A, A)} \wedge \typeRule{P'}{(A, A)}$

\begin{equation}
\label{fDef}
\mbox{let $f = (\lambda pairs. join(\deno{P'}, pairs))$}\end{equation}
$$\mbox{then}$$
\begin{equation} \label{case Exactly(n, P)}\denoRule{(a, b)}{P} \Leftrightarrow (a, b) \in  f^n \deno{Id_A}\end{equation}
$$\mbox{hence it suffices to prove}$$
\begin{equation}
\label{exactly Deno} (a, b)\in f^n \deno{Id_A} \Leftrightarrow \opRule{(a, b)}{(A, A)}{Exactly(n, P')}
\end{equation}

\paragraph{Case $Exactly(0, P'):$}
$$f^0\deno{Id_A} = \deno{Id_A}$$
$$\mbox{so by $\phiRule{(a, b)}{A, A)}{Id_A}$}$$
\begin{equation}
\begin{split}
(a, b) \in f^0\deno{Id_A} & \Leftrightarrow \denoRule{(a,b)}{Id_A}\\
& \Leftrightarrow \opRule{(a, b)}{(A, A)}{Id_A}\\
& \Leftrightarrow \opRule{(a, b)}{(A, A)}{Exactly(0, P')}
\end{split}
\end{equation}

\paragraph{Case $Exactly(n+1, P')$, assuming $\phiRule{Exactly(n, P')}{A}{A}$:}
$$f^{n+1}\deno{Id_A} = f(f^n(\deno{Id_A}))$$
$$\mbox{so by $\phiRule{Exactly(n, P')}{A}{A}$}$$
\begin{equation}
\begin{split}
(a, b) \in f^{n+1}\deno{Id_A} & \Leftrightarrow \exists a'. \denoRule{(a,a')}{P'} \wedge (a', b) \in f^n(\deno{Id_A})\indent\mbox{by definition of $join$ and $f$}\\
& \Leftrightarrow \opRule{(a, a')}{(A, A)}{P} \wedge \opRule{(a', b)}{(A, A)}{Exactly(n, p)}\\
& \Leftrightarrow \opRule{(a, b)}{(A, A)}{Exactly(n+1, P')}\mbox{ by (Exactly n+1)}\\
\end{split}
\end{equation}

\paragraph{Case $P = Upto(n, P')$}
$\\ \indent$We have by inversion of the (Upto) type rule $\typeRule{P}{(A, A)} \wedge \typeRule{P'}{(A, A)}$

\begin{equation}
\label{fDef}
\mbox{let $f = (\lambda pairs. join(\deno{P'}, pairs) \cup pairs)$}\end{equation}
$$\mbox{then}$$

\begin{equation}
\deno{P} = f^n\deno{Id_A}\\
\mbox{We now case split on n}
\end{equation}

\paragraph{Case $Upto(0, P')$}

\begin{equation}
\begin{split}
\denoRule{(a,b)}{Upto(0, P')} & \Leftrightarrow (a, b) \in f^0 \deno{Id_A} \\
							  & \Leftrightarrow (a, b) \in \deno{Id_A}\\ 
							  & \Leftrightarrow \opRule{(a, b)}{(A, A)}{Id_A} \mbox{    by $\phiRule{Id_A}{A}{A}$}\\ 
							  & \Leftrightarrow \opRule{(a, b)}{(A, A)}{Upto(0, P')} \mbox{    by (Upto0)}\\ 
\end{split}
\end{equation}

\paragraph{Case $Upto(n+1, P')$}

\begin{equation}
\begin{split}
\denoRule{(a,b)}{Upto(m + 1, P')} & \Leftrightarrow (a, b) \in f^{m+1} \deno{Id_A} \\
							  & \Leftrightarrow (a, b) \in (join(\deno{P'}, f^{m} \deno{Id_A}) \cup f^{m} \deno{Id_A})\\ 
							  & \Leftrightarrow (a, b) \in join(\deno{P'}, \deno{Upto(m, P)}) \vee (a, b) \in \deno{Upto(m, P')}\\ 
							  & \Leftrightarrow (\exists a'.\denoRule{(a, a')}{P'} \wedge \denoRule{(a', b)}{Upto(m, P')}) \vee \opRule{(a, b)}{(A, A)}{Upto(m, P')}\\ 
							  & \Leftrightarrow (\exists a'.\opRule{(a, a')}{(A, A)}{P'} \wedge \opRule{(a', b)}{(A, A)}{Upto(m, P')}) \vee \opRule{(a, b)}{(A, A)}{Upto(m, P')}\\ 
							  & \Leftrightarrow \opRule{(a, b)}{(A, A)}{Upto(m+1, P')}\mbox{   by (Upto n+1), (Upto n)}
\end{split}
\end{equation}


\paragraph{case $P = FixedPoint(P')$}
$\\ \indent$We have by inversion of the (FixedPoint) type rule $\typeRule{P}{(A, A)} \wedge \typeRule{P'}{(A, A)}$

\begin{equation}
\label{fDef}
\mbox{let $f = (\lambda pairs. join(\deno{P'}, pairs) \cup pairs)$}\end{equation}
$$\mbox{then}$$
$$
\deno{P} = fix(f) \mbox{   In the domain $\clos$}\\
$$
\paragraph{Lemma:  f is continuous in the domain $\clos$}
\setlength{\leftskip}{1cm}
\paragraph{Firstly, f is monotonous.} $\\ $
Let $x \subseteq y$
\begin{equation}
\label{continuousLemma}
\begin{split}
(a, b) \in f(x) & \Rightarrow (a, b) \in x \vee (\exists a'. \denoRule{(a, a')}{P'} \wedge (a', b) \in x)\\
& \Rightarrow (a,b) \in y \vee(\exists a'. \denoRule{(a, a')}{P'} \wedge (a', b) \in y)\\
& \Rightarrow (a, b) \in f(y)\\
\therefore & f(x) \subseteq f(y)
\end{split}
\end{equation}
\setlength{\leftskip}{0pt}


\setlength{\leftskip}{1cm}
\paragraph{Secondly, $f$ preserves the $lub$s of chains.} $\\ $
$\indent\indent$ Consider a chain $x1 \subseteq x2 \subseteq ... $ in $\clos$
Since $\clos$ is a domain, the $lub$, $\bigcup_nx_n$ is also in $\clos$

 
\begin{equation}
\begin{split}
	\forall m. x_m & \subseteq \bigcup_n x_n \\
	\forall m. f(x_n) & \subseteq f(\bigcup_n(x_n))\\
	\therefore \bigcup_nf(x_n) & \subseteq f(\bigcup_n(x_n))
\end{split}
\end{equation}

To get the inverse relation,

\begin{equation}
(a, b) \in f(\bigcup_n(x_n)) \Rightarrow ((\exists n. (a,b) \in x_n ) \vee (\exists m, a'. \denoRule{(a, a')}{P'} \wedge (a', b) \in x_m)
\end{equation}

let $n' = max(n, m)$ so $x_n \subseteq x_{n'}  \wedge x_m \subseteq x_{n'}$

\begin{equation}
\begin{split}
\exists n'. ((a,b) & \in x_{n'} ) \vee (\exists a'. \denoRule{(a, a')}{P'} \wedge (a', b) \in x_{n'})\\
\therefore\exists n'. (a, b) &\in f(x_{n'})\\
\therefore\exists n'. f(\bigcup_nx_n) &\subseteq f(x_{n'}\\
\therefore f(\bigcup_nx_n) &\subseteq \bigcup_nf(x_{n'})\\
\end{split}
\end{equation}


\setlength{\leftskip}{0pt}
So $f$ is Scott-continuous.

Now, by Tarski's fixed point theorem

$$\deno{FixedPoint(P')} = fix(f) = \bigsqcup_nf^n(\bot)$$


\paragraph{Lemma $\opRule{(a, b)}{(A, A)}{FixedPoint(P')} \Leftrightarrow \exists n. (a, b) \in f^n(\bot)$}

Firstly, in the forwards direction, $\opRule{(a, b)}{(A, A)}{FixedPoint(P')} \Rightarrow \exists n. (a, b) \in f^n(\bot)$

By inversion of the operational rules (FixedPoint0), (FixedPoint n) and $\opRule{(a, b)}{(A, A)}{FixedPoint(P')}$, we get two cases.



\paragraph{Case $\opRule{(a,b)}{A, A}{Id_A}$:\\}
by $\phiRule{Id_A}{A}{A}, \denoRule{(a, b)}{Id_A} = \bot$\\so $n = 0$


\paragraph{Case $\opRule{(a, b)}{(A,A)}{P'} \wedge \opRule{(b, c)}{(A,A)}{FixedPoint(P')}\\ $}
(hence $\opRule{(a, c)}{(A, A)}{FixedPoint(P')}$)

by $\phiRule{P'}{A}{A}, and \opRule{(b,c)}{(A,A)}{FixedPoint(P')}$

$$\denoRule{(a, b)}{P'} \wedge \exists n. (b, c) \in f^n(\bot)$$

Instantiating with $m = n$ gives
$$\denoRule{(a, b)}{P'} \wedge (b, c) \in f^m(\bot)$$

so $(a,c) \in f(f^m(\bot)) = f^{m+1}(\bot)$

Hence $\opRule{(a, b)}{(A, A)}{FixedPoint(P')} \Rightarrow \exists n. (a, b) \in f^n(\bot)\\\\ $
To go the other way, we need to prove $(a, b) \in \bigsqcup_nf^n(\bot) \Rightarrow \opRule{(a, b)}{(A, A)}{FixedPoint(P')}$

$(a, b) \in \bigsqcup_n f^n(\bot)$ Means that either:

\paragraph{Case $(a, b) \in \bot$}
\begin{equation}
\label{Case Bottom}
\begin{split}
& \therefore \denoRule{(a,b)}{Id_A}\\
& \therefore \opRule{(a,b)}{(A, A)}{Id_A} \mbox{ By $\phiRule{Id_a}{A}{A}$}\\
& \therefore \denoRule{(a,b)}{Id_A} \mbox{ By (Fix1)}
\end{split}
\end{equation}

\paragraph{Case $\exists n \geq 0. (a, b) \in f^{n+1}(\bot) \wedge \neg ((a, b) \in f^n(\bot))$}
\begin{equation}
\label{Case n+1}
\begin{split}
(\exists a'. \denoRule{(a, a')}{P'} \wedge (a', b) \in f^{n}(\bot)) & \vee (a, b) \in f^{n}(\bot) \wedge \neg ((a, b) \in f^n(\bot))\\
&\therefore \exists a'. \denoRule{(a, a')}{P'} \wedge (a', b) \in f^{n}(\bot)\\
&\therefore \opRule{(a, a')}{(A, A)}{P'} \wedge \opRule{(a', b)}{(A, A)}{FixedPoint(P')}\\
&\therefore \opRule{(a,b)}{(A,A)}{FixedPoint(P')}
\end{split}
\end{equation}

so we have $\opRule{(a, b)}{(A, A)}{FixedPoint(P')} \Leftrightarrow \exists n. (a, b) \in f^n(\bot) \Leftrightarrow \denoRule{(a, b)}{FixedPoin(P')}$

$$\square$$

\subsection{Write Semantics}
We have fairly simple write semantics, we define the type of the $write$ function as mapping a view and a set of pairs related by a relation to a new view.

\begin{equation}
write \colon View_{\Sigma} \rightarrow \wp(A \times R \times B) \rightharpoonup View_{\Sigma} \mbox{For $A, B \in \tau$}
\end{equation}

We define $write$ as so:\\
let $rs = \setComp{(a_i, r_i, b_i) \in (A \times R \times B)}{0 < i \leqslant n}$ for some $n$ being the size of the set.

\begin{equation}
\begin{split}
write(v)(rs) = &v\left[A \mapsto v_{table}(A) \cup \setComp{a_i}{0 < i \leqslant n}\right]\\
& \left[B \mapsto v_{table}(B) \cup \setComp{b_i}{0 < i \leqslant n}\right]\\
& \left[r \mapsto v_{rel}(R) \cup \setComp{(a_i, b_i)}{0 < i \leqslant n}\right] 
\end{split}
\end{equation}

\end{document}